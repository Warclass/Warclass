\section{RESUMEN}

El presente proyecto surge como respuesta al desalineamiento existente entre las tareas educativas y la motivación de los estudiantes en contextos secundarios, institucionales y universitarios. En el sistema educativo actual, los profesores asignan tareas de investigación que muchos estudiantes perciben como poco útiles para su vida real, lo que provoca que prioricen aprobar antes que aprender genuinamente \cite{deci2017}. Ante esta percepción de irrelevancia, los alumnos recurren a atajos digitales como copiar contenido de internet, transcribir sin analizar, o usar herramientas de inteligencia artificial que proporcionan respuestas inmediatas sin fomentar el proceso crítico \cite{chen2023}. Esta problemática genera un aprendizaje superficial donde las tareas se cumplen sin una comprensión real del contenido, desaprovechando el potencial educativo de las herramientas tecnológicas disponibles \cite{martinez2023}.

Se propone una aplicación educativa 3D para la gamificación de tareas. La solución consiste en una aplicación para docentes y estudiantes, donde acceden a un mundo virtual medieval gamificando las actividades académicas en misiones y aventuras. Este ecosistema educativo digital automatiza el ciclo de enseñanza-aprendizaje desde la creación de contenidos hasta la evaluación y seguimiento del progreso de cada estudiante.

El desarrollo implementará componentes como autenticación segura, entornos 3D virtuales inmersivos, sistemas de misiones y recompensas, evaluación integrada y análisis de progreso. La plataforma incorporará tecnologías de inteligencia artificial para personalización del aprendizaje. Adicionalmente, se integrará con plataformas educativas existentes como Canvas, Discord y otras herramientas digitales para crear un ecosistema educativo completo.

\textbf{Palabras clave:} gamificación educativa, entorno virtual 3D, aprendizaje interactivo, motivación estudiantil, personalización educativa, aplicación móvil, análisis de aprendizaje, educación secundaria, educación universitaria.

\section{CAPÍTULO IV: Desarrollo del proyecto}

\subsection{4.1 Metodología de desarrollo}

Para el desarrollo de la plataforma de gamificación educativa se adoptó la metodología Kanban, seleccionada por su flexibilidad para equipos pequeños y su capacidad de adaptación a flujos de trabajo variables. Esta metodología permite gestionar el desarrollo iterativo mediante un tablero visual que facilita el seguimiento del progreso y la identificación de cuellos de botella.

\subsubsection{4.1.1 Configuración del tablero Kanban}

El tablero Kanban se estructuró en las siguientes columnas para optimizar el flujo de trabajo del proyecto:

\begin{itemize}
\item \textbf{Backlog}: Tareas identificadas y priorizadas pendientes de inicio
\item \textbf{En análisis}: Tareas en proceso de definición técnica y diseño
\item \textbf{En desarrollo}: Implementación activa de funcionalidades
\item \textbf{En pruebas}: Validación y testing de componentes desarrollados  
\item \textbf{En revisión}: Revisión de código y documentación
\item \textbf{Finalizado}: Tareas completadas e integradas al sistema
\end{itemize}

\begin{figure}[H]
	\centering
	\includegraphics[width=0.85\textwidth]{images/kanban.png}
	\caption{Configuración del tablero Kanban para el desarrollo del proyecto (Figura 14).}
	\label{fig:kanban-board}
\end{figure}

\subsubsection{4.1.2 Límites de trabajo en progreso (WIP)}

Se establecieron límites WIP para mantener la eficiencia del flujo y evitar la sobrecarga de trabajo:

\begin{itemize}
\item \textbf{En análisis}: Máximo 3 elementos
\item \textbf{En desarrollo}: Máximo 4 elementos  
\item \textbf{En pruebas}: Máximo 2 elementos
\item \textbf{En revisión}: Máximo 2 elementos
\end{itemize}

\begin{figure}[H]
	\centering
	\includegraphics[width=0.85\textwidth]{images/metricas.png}
	\caption{Métricas de rendimiento del flujo Kanban durante el desarrollo (Figura 15).}
	\label{fig:kanban-metrics}
\end{figure}

\subsection{4.2 Entregables del proyecto}

El desarrollo se organizó en iteraciones incrementales, con entregables específicos que aportan valor funcional al sistema. Cada entregable fue gestionado a través del tablero Kanban, desde su conceptualización hasta su implementación completa.

\subsubsection{4.2.1 Sprint 0: Configuración inicial}

\paragraph{Entregables}
\begin{itemize}
\item Configuración del entorno de desarrollo (Node.js, PostgreSQL, Docker)
\item Estructura base del proyecto con Next.js y TypeScript
\item Configuración de Prisma ORM y esquema inicial de base de datos
\item Pipeline de CI/CD básico
\end{itemize}

\begin{figure}[H]
	\centering
	\includegraphics[width=0.85\textwidth]{images/arquitectura.png}
	\caption{Configuración del entorno de desarrollo del proyecto (Figura 16).}
	\label{fig:dev-environment}
\end{figure}

\subsubsection{4.2.2 Sprint 1: Sistema de autenticación}

\paragraph{Entregables}
\begin{itemize}
\item Implementación de autenticación JWT con NextAuth.js
\item Integración con proveedores OAuth2 (Google, Microsoft)
\item Sistema de roles y permisos (estudiante, docente, administrador)
\item Interfaces de login y registro de usuarios
\end{itemize}

\begin{figure}[H]
	\centering
	\includegraphics[width=0.85\textwidth]{images/pagina_web_iniciar-sesion.png}
	\caption{Diagrama de flujo del sistema de autenticación implementado (Figura 17).}
	\label{fig:auth-flow}
\end{figure}

\subsubsection{4.2.3 Sprint 2: Gestión de usuarios y perfiles}

\paragraph{Entregables}
\begin{itemize}
\item CRUD completo para gestión de usuarios
\item Sistema de perfiles con avatares personalizables
\item Dashboard diferenciado por rol de usuario
\item Configuraciones de privacidad y notificaciones
\end{itemize}

\begin{figure}[H]
	\centering
	\includegraphics[width=0.85\textwidth]{images/pagina_web_panel-administrativo.png}
	\caption{Panel de administración para gestión de usuarios y perfiles (Figura 18).}
	\label{fig:user-management}
\end{figure}

\subsubsection{4.2.4 Sprint 3: Motor de gamificación}

\paragraph{Entregables}
\begin{itemize}
\item Sistema de puntos y niveles
\item Gestión de logros e insignias
\item Algoritmo de progresión basado en actividades
\item API para tracking de eventos de gamificación
\end{itemize}

\begin{figure}[H]
	\centering
	\includegraphics[width=0.85\textwidth]{images/pagina_web_gestion-de-evaluaciones.jpg}
	\caption{Arquitectura del motor de gamificación y sistema de progresión (Figura 19).}
	\label{fig:gamification-engine}
\end{figure}

\subsubsection{4.2.5 Sprint 4: Entorno 3D con Three.js}

\paragraph{Entregables}
\begin{itemize}
\item Integración de Three.js en el frontend React
\item Escena 3D básica con mundo medieval
\item Sistema de avatares tridimensionales
\item Controles de navegación e interacción
\end{itemize}

\begin{figure}[H]
	\centering
	\includegraphics[width=0.85\textwidth]{images/pagina_web_vista-alumno.png}
	\caption{Vista del entorno virtual 3D implementado con Three.js (Figura 20).}
	\label{fig:3d-environment}
\end{figure}

\subsubsection{4.2.6 Sprint 5: Sistema de misiones educativas}

\paragraph{Entregables}
\begin{itemize}
\item Creador de misiones para docentes
\item Sistema de asignación automática y manual
\item Tracking de progreso en tiempo real
\item Integración con el motor de gamificación
\end{itemize}

\begin{figure}[H]
	\centering
	\includegraphics[width=0.85\textwidth]{images/pagina_web_crear-una-clase.jpg}
	\caption{Sistema de creación y gestión de misiones educativas (Figura 21).}
	\label{fig:mission-creator}
\end{figure}

\subsection{4.3 Gestión de riesgos y contingencias}

Durante el desarrollo se implementaron estrategias de mitigación de riesgos basadas en la flexibilidad de Kanban para adaptar el flujo de trabajo ante contingencias.

\subsubsection{4.3.1 Identificación de riesgos técnicos}

La siguiente matriz identifica los principales riesgos técnicos del proyecto, evaluando su probabilidad de ocurrencia, impacto potencial y estrategias de mitigación implementadas:

\begin{table}[H]
	\centering
	\caption{Matriz de análisis de riesgos técnicos del proyecto.}
	\label{tab:risk-matrix}
	\begin{tabular}{p{3.5cm}p{1.8cm}p{1.8cm}p{5.5cm}}
		\toprule
		\textbf{Riesgo} & \textbf{Prob.} & \textbf{Impacto} & \textbf{Estrategia de Mitigación} \\
		\midrule
		Rendimiento del entorno 3D & Alta & Alto & Optimización de mallas y texturas, implementación de LOD (Level of Detail), occlusion culling, lazy loading de assets, uso de texture atlases \\
		\midrule
		Escalabilidad de la base de datos & Media & Alto & Implementación de índices optimizados, consultas con joins eficientes, uso de Prisma ORM para prevenir N+1 queries, conexiones pooling, cacheo con Redis \\
		\midrule
		Compatibilidad cross-browser & Media & Medio & Testing automatizado en Chrome, Firefox, Safari y Edge, uso de polyfills, transpilación con Babel, validación de WebGL 2.0 support \\
		\midrule
		Seguridad de autenticación & Baja & Crítico & Implementación de JWT con tokens de corta duración, refresh tokens seguros, OAuth2 con proveedores confiables, hashing bcrypt, sanitización de inputs, HTTPS obligatorio \\
		\midrule
	\end{tabular}
\end{table}

\paragraph{Clasificación de probabilidad e impacto}
\begin{itemize}
	\item \textbf{Probabilidad}: Baja (< 25\%), Media (25-50\%), Alta (> 50\%)
	\item \textbf{Impacto}: Bajo (afecta funcionalidad menor), Medio (afecta experiencia de usuario), Alto (afecta funcionalidad crítica), Crítico (compromete seguridad o integridad de datos)
\end{itemize}

\subsection{4.4 Control de versiones y colaboración}

Se estableció un flujo de trabajo con Git que facilita la colaboración y el control de versiones del código fuente.

\subsubsection{4.4.1 Estrategia de branching}

\begin{itemize}
\item \textbf{main}: Rama principal con código estable en producción
\item \textbf{develop}: Rama de desarrollo con últimas características integradas
\item \textbf{stage}: Rama de staging para pruebas previas a producción
\end{itemize}

\begin{figure}[H]
	\centering
	\includegraphics[width=0.85\textwidth]{images/branching.png}
	\caption{Estrategia de branching y flujo de trabajo con Git (Figura 23).}
	\label{fig:git-workflow}
\end{figure}

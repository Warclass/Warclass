\section{CAPÍTULO IV: Desarrollo del proyecto}

\subsection{4.1 Metodología de desarrollo}

Para el desarrollo de la plataforma de gamificación educativa se adoptó la metodología Kanban, seleccionada por su flexibilidad para equipos pequeños y su capacidad de adaptación a flujos de trabajo variables. Esta metodología permite gestionar el desarrollo iterativo mediante un tablero visual que facilita el seguimiento del progreso y la identificación de cuellos de botella.

\subsubsection{4.1.1 Configuración del tablero Kanban}

El tablero Kanban se estructuró en las siguientes columnas para optimizar el flujo de trabajo del proyecto:

\begin{itemize}
\item \textbf{Backlog}: Tareas identificadas y priorizadas pendientes de inicio
\item \textbf{En análisis}: Tareas en proceso de definición técnica y diseño
\item \textbf{En desarrollo}: Implementación activa de funcionalidades
\item \textbf{En pruebas}: Validación y testing de componentes desarrollados  
\item \textbf{En revisión}: Revisión de código y documentación
\item \textbf{Finalizado}: Tareas completadas e integradas al sistema
\end{itemize}

\begin{figure}[H]
	\centering
	\fbox{\parbox{0.8\textwidth}{\centering \textbf{Placeholder: Tablero Kanban del Proyecto}\\ \vspace{1cm} Figura mostrando la configuración del tablero Kanban con las columnas definidas y ejemplos de tarjetas de trabajo en diferentes estados del flujo.}}
	\caption{Configuración del tablero Kanban para el desarrollo del proyecto (Figura 14).}
	\label{fig:kanban-board}
\end{figure}

\subsubsection{4.1.2 Límites de trabajo en progreso (WIP)}

Se establecieron límites WIP para mantener la eficiencia del flujo y evitar la sobrecarga de trabajo:

\begin{itemize}
\item \textbf{En análisis}: Máximo 3 elementos
\item \textbf{En desarrollo}: Máximo 4 elementos  
\item \textbf{En pruebas}: Máximo 2 elementos
\item \textbf{En revisión}: Máximo 2 elementos
\end{itemize}

\begin{figure}[H]
	\centering
	\fbox{\parbox{0.8\textwidth}{\centering \textbf{Placeholder: Métricas de Flujo Kanban}\\ \vspace{1cm} Gráfico mostrando el tiempo de ciclo, throughput y distribución del trabajo a lo largo del desarrollo del proyecto.}}
	\caption{Métricas de rendimiento del flujo Kanban durante el desarrollo (Figura 15).}
	\label{fig:kanban-metrics}
\end{figure}

\subsection{4.2 Entregables del proyecto}

El desarrollo se organizó en iteraciones incrementales, con entregables específicos que aportan valor funcional al sistema. Cada entregable fue gestionado a través del tablero Kanban, desde su conceptualización hasta su implementación completa.

\subsubsection{4.2.1 Sprint 0: Configuración inicial}

\paragraph{Entregables}
\begin{itemize}
\item Configuración del entorno de desarrollo (Node.js, PostgreSQL, Docker)
\item Estructura base del proyecto con Next.js y TypeScript
\item Configuración de Prisma ORM y esquema inicial de base de datos
\item Pipeline de CI/CD básico
\end{itemize}

\begin{figure}[H]
	\centering
	\fbox{\parbox{0.8\textwidth}{\centering \textbf{Placeholder: Arquitectura del Entorno de Desarrollo}\\ \vspace{1cm} Diagrama mostrando la configuración del entorno local y cloud, incluyendo bases de datos, servicios y herramientas de desarrollo.}}
	\caption{Configuración del entorno de desarrollo del proyecto (Figura 16).}
	\label{fig:dev-environment}
\end{figure}

\subsubsection{4.2.2 Sprint 1: Sistema de autenticación}

\paragraph{Entregables}
\begin{itemize}
\item Implementación de autenticación JWT con NextAuth.js
\item Integración con proveedores OAuth2 (Google, Microsoft)
\item Sistema de roles y permisos (estudiante, docente, administrador)
\item Interfaces de login y registro de usuarios
\end{itemize}

\begin{figure}[H]
	\centering
	\fbox{\parbox{0.8\textwidth}{\centering \textbf{Placeholder: Flujo de Autenticación}\\ \vspace{1cm} Diagrama de secuencia mostrando el proceso de autenticación desde el login hasta la obtención del token JWT y acceso a recursos protegidos.}}
	\caption{Diagrama de flujo del sistema de autenticación implementado (Figura 17).}
	\label{fig:auth-flow}
\end{figure}

\subsubsection{4.2.3 Sprint 2: Gestión de usuarios y perfiles}

\paragraph{Entregables}
\begin{itemize}
\item CRUD completo para gestión de usuarios
\item Sistema de perfiles con avatares personalizables
\item Dashboard diferenciado por rol de usuario
\item Configuraciones de privacidad y notificaciones
\end{itemize}

\begin{figure}[H]
	\centering
	\fbox{\parbox{0.8\textwidth}{\centering \textbf{Placeholder: Dashboard de Gestión de Usuarios}\\ \vspace{1cm} Captura de pantalla del panel administrativo mostrando la gestión de usuarios, asignación de roles y configuración de perfiles.}}
	\caption{Panel de administración para gestión de usuarios y perfiles (Figura 18).}
	\label{fig:user-management}
\end{figure}

\subsubsection{4.2.4 Sprint 3: Motor de gamificación}

\paragraph{Entregables}
\begin{itemize}
\item Sistema de puntos y niveles
\item Gestión de logros e insignias
\item Algoritmo de progresión basado en actividades
\item API para tracking de eventos de gamificación
\end{itemize}

\begin{figure}[H]
	\centering
	\fbox{\parbox{0.8\textwidth}{\centering \textbf{Placeholder: Arquitectura del Motor de Gamificación}\\ \vspace{1cm} Diagrama de componentes mostrando el motor de gamificación, sistema de puntos, logros y su integración con el sistema académico.}}
	\caption{Arquitectura del motor de gamificación y sistema de progresión (Figura 19).}
	\label{fig:gamification-engine}
\end{figure}

\subsubsection{4.2.5 Sprint 4: Entorno 3D con Three.js}

\paragraph{Entregables}
\begin{itemize}
\item Integración de Three.js en el frontend React
\item Escena 3D básica con mundo medieval
\item Sistema de avatares tridimensionales
\item Controles de navegación e interacción
\end{itemize}

\begin{figure}[H]
	\centering
	\fbox{\parbox{0.8\textwidth}{\centering \textbf{Placeholder: Entorno 3D Medieval}\\ \vspace{1cm} Captura del mundo virtual 3D mostrando el ambiente medieval, avatares de estudiantes y elementos interactivos de gamificación.}}
	\caption{Vista del entorno virtual 3D implementado con Three.js (Figura 20).}
	\label{fig:3d-environment}
\end{figure}

\subsubsection{4.2.6 Sprint 5: Sistema de misiones educativas}

\paragraph{Entregables}
\begin{itemize}
\item Creador de misiones para docentes
\item Sistema de asignación automática y manual
\item Tracking de progreso en tiempo real
\item Integración con el motor de gamificación
\end{itemize}

\begin{figure}[H]
	\centering
	\fbox{\parbox{0.8\textwidth}{\centering \textbf{Placeholder: Creador de Misiones}\\ \vspace{1cm} Interfaz del sistema de creación de misiones mostrando formularios, configuración de objetivos y asignación de recompensas.}}
	\caption{Sistema de creación y gestión de misiones educativas (Figura 21).}
	\label{fig:mission-creator}
\end{figure}

\subsection{4.3 Gestión de riesgos y contingencias}

Durante el desarrollo se implementaron estrategias de mitigación de riesgos basadas en la flexibilidad de Kanban para adaptar el flujo de trabajo ante contingencias.

\subsubsection{4.3.1 Identificación de riesgos técnicos}

\begin{itemize}
\item \textbf{Rendimiento del entorno 3D}: Optimización de mallas y texturas
\item \textbf{Escalabilidad de la base de datos}: Implementación de índices y consultas optimizadas  
\item \textbf{Compatibilidad cross-browser}: Testing en múltiples navegadores
\item \textbf{Seguridad de autenticación}: Implementación de buenas prácticas de seguridad
\end{itemize}

\begin{figure}[H]
	\centering
	\fbox{\parbox{0.8\textwidth}{\centering \textbf{Placeholder: Matriz de Riesgos del Proyecto}\\ \vspace{1cm} Matriz mostrando la identificación, probabilidad, impacto y estrategias de mitigación de los principales riesgos técnicos del proyecto.}}
	\caption{Matriz de análisis de riesgos y estrategias de mitigación (Figura 22).}
	\label{fig:risk-matrix}
\end{figure}

\subsection{4.4 Control de versiones y colaboración}

Se estableció un flujo de trabajo con Git que facilita la colaboración y el control de versiones del código fuente.

\subsubsection{4.4.1 Estrategia de branching}

\begin{itemize}
\item \textbf{main}: Rama principal con código estable en producción
\item \textbf{develop}: Rama de desarrollo con últimas características integradas
\item \textbf{feature/*}: Ramas para desarrollo de funcionalidades específicas
\item \textbf{hotfix/*}: Ramas para correcciones críticas en producción
\end{itemize}

\begin{figure}[H]
	\centering
	\fbox{\parbox{0.8\textwidth}{\centering \textbf{Placeholder: Flujo de Git Branching}\\ \vspace{1cm} Diagrama mostrando la estrategia de ramificación Git utilizada, con ejemplos de merge requests y flujo de integración continua.}}
	\caption{Estrategia de branching y flujo de trabajo con Git (Figura 23).}
	\label{fig:git-workflow}
\end{figure}
